\section{Proofs of Propositions}
\subsection*{Preliminaries}
We use: (i) stability of linear recurrences; (ii) Lipschitz continuity of $x\mapsto a/x$ on $[\underline{I},\infty)$; (iii) ratio perturbation bound $|(a+\Delta a)/(b+\Delta b)-a/b| \le \frac{|\Delta a|}{|b|} + \frac{|a|}{|b|^2}|\Delta b|$.

\begin{proof}[Proof of \Cref{prop:screening}]
Under \Cref{ass:growth,ass:margins,ass:sweep,ass:rates,ass:cpi,ass:bs}, we establish bounds on each error component:

\textbf{Step 1: Debt evolution error.} From the linear recurrence $D_{t+1} = (1+r_d)D_t - s \cdot \text{FCF}_t$, solving exactly and bounding FCF linearization error over the perturbation domain $(\delta g, \delta \kappa) \in [-0.02, 0.02]^2$ gives:
\begin{align}
\epsilon_D(t) &\leq (1+r_d)^t \epsilon_D(0) + \frac{s}{|(1+r_d) - (1+g)|} C_{\mathrm{FCF}} \cdot \text{EBITDA}_0 |(1+g)^t - (1+r_d)^t|
\end{align}
where $C_{\mathrm{FCF}}$ derives from $(\overline{\varphi},\overline{\kappa},\overline{e},g_{\max})$ and we use the limit form $t(1+r_d)^{t-1}$ when $r_d = g$.

\textbf{Step 2: Lease schedule error.} Under CPI uncertainty range $[\underline{\pi}, \overline{\pi}]$ with monotonicity:
\begin{align}
\epsilon_L(t) &\leq L_0 \frac{(\overline{\pi} - \underline{\pi}) t (1 + \overline{\pi})^{t-1}}{1 - (1 + \underline{\pi})^{-T}}
\end{align}

\textbf{Step 3: EBITDA linearization.} From first-order Taylor remainder bounds:
\begin{align>
\epsilon_{\text{EBITDA}}(t) &\leq \text{EBITDA}_t (|\delta g| \cdot t + |\delta \kappa|)
\end{align>

\textbf{Step 4: Ratio bounds.} Using ratio perturbation inequality and ex-ante lower bound $\underline{I}_t > 0$:
\begin{align}
\epsilon_{\text{ICR}}(t) &= \frac{\text{EBITDA}_t}{(\underline{I}_t)^2}\bigl(r_d\,\epsilon_D(t)+r_L\,\epsilon_L(t)\bigr) + \frac{\epsilon_{\text{EBITDA}}(t)}{\underline{I}_t} \\
\epsilon_{\text{Lev}}(t) &= \frac{\epsilon_D(t)+\epsilon_L(t)}{\text{EBITDA}_t} + \frac{\text{Net Debt}_t\,\epsilon_{\text{EBITDA}}(t)}{\text{EBITDA}_t^2}
\end{align}
\end{proof}

\begin{proof}[Proof of \Cref{prop:frontier}]
Set inclusion follows from relaxation: increasing $c^{\text{lev}}$ from $c^{\text{lev}}_1$ to $c^{\text{lev}}_2$ enlarges the feasible constraint set $\mathcal{F}(c^{\text{lev}}_1) \subseteq \mathcal{F}(c^{\text{lev}}_2)$. Since the objective $\E[\text{IRR}(\mathcal{C};\theta)]$ is non-decreasing in the size of the feasible set, monotonicity follows.
\end{proof}

\begin{proof}[Proof of \Cref{prop:certification}]
If $h_{\text{analytic}}(t) > \epsilon_{\max}$ for all $t$, then the analytic approximation provides sufficient margin that even under worst-case error, true ratios remain within covenant thresholds. Specifically:
\begin{align}
\text{ICR}_{\text{true}}(t) &\geq \text{ICR}_{\text{analytic}}(t) - \epsilon_{\text{ICR}}(t) \geq c^{\text{icr}} + h_{\text{analytic}}(t) - \epsilon_{\max} > c^{\text{icr}} \\
\text{Leverage}_{\text{true}}(t) &\leq \text{Leverage}_{\text{analytic}}(t) + \epsilon_{\text{Lev}}(t) \leq c^{\text{lev}} - h_{\text{analytic}}(t) + \epsilon_{\max} < c^{\text{lev}}
\end{align}
Thus feasibility is certified with certainty under our approximation assumptions.
\end{proof>

\paragraph{Tightness and counterexample.}
The bounds are constructively tight: when CPI drifts at upper bound $\overline{\pi}$ monotonically, $\epsilon_L(t)$ approaches the stated bound. In the loose regime when growth volatility exceeds 12\% annually, Taylor approximations become conservative but may overestimate error margins by 20-30\%.

\paragraph{Failure modes.}
Certification fails when $\underline{I}_t \to 0$ (near-zero interest expense) or $\text{EBITDA}_t \to \underline{e}$ (margin compression). In such cases, the method reverts to Monte Carlo simulation without analytic guarantees.
